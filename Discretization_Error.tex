%Discretization Error
\section{Discretization Error}
The accuracy, or discretization error, of a piecewise, linear 
representation ($discretization$) of an analytical curve ($curve$) in 
$R^3$ can be defined in many ways depending on the intended application.  
The error associated with the $discretization$ is discussed in terms of 
the ``deviation'' from the $curve$ -- most often quantified by calculating 
or approximating the distance from the $curve$ for each linear segment in 
the $discretization$, or the area of the ruled surface between a 
curve-piece and the segment representing that part of the $curve$.  
Another way of quantifying the error associated with a $discretization$ 
would be to consider how well it approximates the arc length of the 
$curve$ it represents.  In general the arc length is not known {\it{a 
priori}}, 
but depending on the underlying representation it can be calculated 
exactly (parametric or analytical) or can be estimated (Bezier).  One of 
the goals of this method was to be ``general'' in that it should be 
independent of the underlying geometric representation.  Therefore, a 
method that requires the arc length of the underlying geometry violates 
the 
aforementioned concept of ``generality'' and restricts the applications 
for which the proposed method could be applied.  Some other way of 
determining/generating an edge grid based on arc length is needed.  This 
process will be detailed later.

Arc-length convergence of a $discretization$ is a sufficient condition for 
other schemes of edge grid generation/refinement.  That is: if the 
difference between the arc length of the $curve$ and the sum of the 
segments in the $discretization$ approaches zero then that is sufficient 
to conclude that the distance between the $discretization$ and the $curve$ 
is also approaching zero, also the angles between segments approaches 180 
degrees.  However, the converse of that statement is not true.  The 
pathological case of a highly oscillatory, low amplitude $curve$ 
approximated by two straight lines (sine-wave approximated by straight 
lines) shows that a $discretization$ of a $curve$ can have a small 
``deviation'' or angles between segments but be a poor estimate for arc 
length.  Another pathological case is a ``nonconvex'' $curve$ where the 
parameterization goes well ``outside'' of the segment.
