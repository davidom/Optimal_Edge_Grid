% Introduction Document
\section{Introduction}
Computational design and analysis has become a fundamental part of 
industry and academia for use in research, development, and manufacturing. 
In general, the accuracy of a computational analysis depends heavily on 
the fidelity of the computational representation of a real-world object or 
phenomenon.  However, the task of creating high fidelity models of an 
actual geometry can be time-consuming--sometimes consuming up to 
seventy-five percent of the time required to produce a solution 
\cite{bischoff05}. This work seeks to improve the process of creating a 
computational model of an object of interest by accelerating the process 
of mesh generation. In general, a valid volume grid (three-dimensional) is 
bounded by surface grids (two-dimensional); surface grids are bounded by edge grids (one-dimensional). At the start of the grid generation hierarchy are point spacing values at the end points of analytical or parametric curves which bound the edge grids. Once the bounding surface grid is generated, volume grid generation is, in most cases, a highly automated process. The same generalization can be made for surface grids and edge grids. 
Algorithms that use automated point creation/insertion for mesh generation of one-, two-, and three-dimensional geometries are ubiquitous~\cite{cubit,delaunay,aflr}. However the authors are only aware of one such automated scheme for setting point spacing values for more automated mesh generation \cite{mclaurin12}.

The computation involved with edge grid generation is trivial when 
compared to volume grid generation -- even with high-order NURBS curves.  
However, the point spacing values at the end points of curves have to be 
set manually in order to satisfy a desired length scale. This manual 
process is time consuming. If geometry repair (gluing, trimming, etc…) is 
not considered, the amount of user input required to generate a volume 
grid can be concentrated on the lowest levels of the grid generation 
hierarchy -- i.e., edge grid generation.  In addition, if the edge grids 
are not generated appropriately then the errors present there, such as 
overly dense or sparse spacing, will be propagated up the grid generation hierarchy and be present in each subsequent higher-dimensional entity.

The proposed algorithm is a general-use method that can be applied to any ``digital curve'' regardless of its 
representation. This is due to every step being developed {\it without} the use of derivatives. Most other methods 
operate on a specific type of curve, such as NURBS or B-splines, and use the specific information available for 
the type of curve in use.  NURBS curves are the de-facto standard in CAD; however, in other fields, such as 
pattern-recognition, other types of digital curves, such as parametric, are more 
common~\cite{interactive_curve_modeling}.  T-splines are also becoming more popular in isogeometric 
analysis, for example~\cite{iga}. 

The justification for the development of these methods lies in the need 
for an automated way of setting point spacing values on curves. Therefore, 
a general algorithm that does not require derivative information to generate a suitable edge grid has been 
developed. A result of not using derivatives is that each step in the algorithm is robust to large changes in 
derivatives or curves that are not ``well behaved'', e.g., they were highly oscillatory. This process can only be 
automated if some way of judging ``how well'' an edge grid represents a curve is present. To this end, a method of 
generating edge grids through constrained optimization is detailed below. Further discussion of element quality, 
robustness, and a framework for implementing the information associated with an optimal edge grid into an existing 
grid generator is also presented. Generating edge grids in a more automated fashion accelerates the process of 
surface grid generation---and ultimately volume grid generation. Using our algorithm, or another automated method 
for setting point spacings does not change the number of steps required for grid generation. However, it does 
reduce the number of manual steps involved in starting the process.

