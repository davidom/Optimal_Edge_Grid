% Element Quality
\section{Element Quality}
A topic that has not been discussed thus far is element quality.  Element 
quality is an intuitive measure for surface elements, e.g. triangles; but 
how to define quality for a line segment?  In this context, the notion of 
quality refers to the change in length between two topologically adjacent 
segments.  For the purposes of representing the curve to a desired 
tolerance, grid quality is not a concern.  However, for most applications 
the grid quality directly affects downstream uses, e.g. CFD, CEM, etc... 
Therefore, some discussion of this topic is warranted.  Grid quality in 
terms of edge grids is most often defined in terms of growth ratio, i.e. 
the length of a segment divided by the length of a topologically adjacent 
segment.  If the growth ratio strongly deviates from unity then the sizes 
of neighboring elements are not similar.  This will lead to a large 
disparity in the size of surface elements generated from them and that 
``error'' in the form of a large growth ratio will be present in both the 
surface grid and volume grid in the vicinity of that portion of the edge 
grid.  This kind of disparate length scales can cause problems for finite 
element methods, finite volume methods, etc...

Traditionally edge grids are generated with an {\it{a priori}} defined 
growth ratio along with other parameters that ensure that the grid has 
good quality.  A priori quality measures can be included in the 
constraints used in the global optimization procedure in a similar fashion 
to how the minimum and maximum edge lengths were included.  
{\it{A posteriori}} methods for quality control could include some type of 
smoothing or (other type of) optimization [find more methods].  The 
authors acknowledge that the set of constraints grows very quickly in number with respect to the 
number of grid points--albeit linearly.  {\bf{Suzanne:  Move with 
definition of optimization problem.}}  There is the set that defines 
the topology (1), the set that defines the minimum (2) and maximum (3) 
edge lengths, and now two more for minimum (4) and maximum (5) growth 
rates--for a total of five times the number of points.  For the local 
procedure, minimum and maximum growth rates could be include by splitting an edge if 
the growth rate is too large or small.

One final method is to use the output from the developed methods, 
the ``optimal'' grid, as input for a grid generator—which presumably has 
strict quality control measures in place.  This would be accomplished by 
using the point spacing values present at the end points of the 
discretization at the end points of the curve.  The resulting edge grid 
from the grid generator could then be analyzed for the purpose of 
determining how far it deviates from ``optimality'' in the interior of the 
curve.  For instance, given a node on a generated edge grid, $x_i$, the 
corresponding parameterized value of $u_i$ could be found.  A point 
spacing value could be calculated at $u_i$ and compared to an interpolated 
value on the ``optimal'' edge grid.  If the values are found to be too 
disparate, then a point spacing source could be added to adjust the point 
spacing at that location.  The point spacing source would control the 
spacing during grid generation so that the generated edge grid would more 
closely resemble the spacing values present in the ``optimal'' grid while 
maintaining the quality typically associated with grid generators. In 
practice this usually leads to generated grids that are more fine than 
required by precision bounds due to the quality constraints imposed during 
grid generation.

