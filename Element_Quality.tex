% Element Quality
\section{Growth Ratio}
Our goal was to develop an automated edge grid generator that accurately 
represents the underlying geometry.  For this expressed purpose of 
representing the curve to a desired tolerance, the grid quality 
is not a concern.  However, for most applications, the grid quality 
directly affects downstream analyses.
% e.g., CFD, or CEM.

Therefore, we explain how to modify our optimization problem in order to 
yield an edge grid that both accurately represents the underlying geometry 
and is of sufficient quality.  Edge grid quality is typically defined in 
terms of the growth ratio, i.e., the length of a segment divided by the 
length of a topologically adjacent segment.  If the growth ratio strongly 
deviates from unity, the sizes of neighboring elements are not 
similar which leads to a large disparity in the sizes of the surface and 
volume elements which are generated.  Such disparate length scales can 
cause problems for numerical partial differential equation methods.

Traditionally edge grids are generated with an {\it{a priori}} defined 
growth ratio along with other parameters that ensure that the grid has 
good quality.  Upper and lower bounds can be included for the grid 
quality (growth ratio) as nonlinear constraints in the optimization problem.  However, in 
this case, the number of constraints grows very quickly, albeit 
linear in the number of grid points.  Alternatively, minimum and maximum 
growth rates could be enforced by the optimization procedure by splitting 
an edge if the growth rate is too large or small.  {\it{A posteriori}} 
methods for quality control could include some type of 
smoothing or optimization~\cite{lori_pat,feasnewt,multiobjective,logbarrier}.

One final method is to use the output from the edge grid generator, the 
``optimal'' grid, as input for a grid generator, which presumably has 
strict quality control measures in place.  This would be accomplished by 
using the point spacing values present at the end points of the 
discretization at the end points of the curve.  The resulting edge grid 
from the grid generator could then be analyzed for the purpose of 
determining how far it deviates from ``optimality'' in the interior of the 
curve.  
%For instance, given a node on a generated edge grid, $x_i$, the 
%corresponding parameterized value of $u_i$ could be found.  A point 
%spacing value could be calculated at $u_i$ and compared to an 
%interpolated 
%value on the ``optimal'' edge grid.  If the values are found to be too 
%disparate, then a point spacing source could be added to adjust the point 
%spacing at that location.  The point spacing source would control the 
%spacing during grid generation so that the generated edge grid would more 
%closely resemble the spacing values present in the ``optimal'' grid while 
%maintaining the quality typically associated with grid generators. In 
%practice this usually leads to generated grids that are more fine than 
%required by precision bounds due to the quality constraints imposed 
%during grid generation.
If the deviation is too large, a point spacing source could be inserted to 
adjust and control the point spacing as desired during grid generation.
