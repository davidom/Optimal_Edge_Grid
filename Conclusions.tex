% Conclusions
\section{Conclusions and Future Work}
%- Paragraph repeating abstract and confirming what was "promised".
% -reduce need for manual intervention - not many knobs and buttons
% -local optimization results in good representation wrt arc-length deficit
% -robustness, accuracy
%- adding {\it{a priori}} grid quality constraints to optimization problem \\
%- global optimization \\
%- surface optimization
An algorithm for edge grid discretization through local optimization was 
developed. In an effort to accelerate the process of grid generation, 
minimal user input is required for the developed method: a single 
parameter which is used as a limit for local refinement. The results show 
that the generated edge grid is optimal with respect to arc-length 
deficit. In addition, the process was shown to be robust to 
discontinuities, abrupt changes in curvature, and self-intersections. 
Results were shown here in two dimensions for ease of presentation. The 
developed algorithm is easily abstracted to three dimensional curves
through a the kernel for edge length calculation.

Future work would include a comparison to a global optimization problem 
formulated with the presented constraints and an additional constraint of 
a given number of edge grid points. Grid quality measures could also be 
included in the optimization problem via {\it a priori} quality 
constraints.

More work will also be done to abstract the problem from strictly 
one-dimensional simplices (edge grids) to two-dimensional simplices 
(triangles). While it was straightforward to determine which part of a 
curve an edge grid represents, it is non-trivial to determine which part 
of a surface a triangle represents. The development of a map between the 
planar elements representing a surface and the underlying geometry would 
be one of the chief tasks moving forward. Additionally, the edges, as well 
as the triangles, in the discretization must considered when optimizing 
the surface grid.

Finally, we will apply our edge and surface grid generation routines on 
problems stemming from real-world applications, including those from 
mechanical engineering and medicine.  One challenge that will need to be 
faced is the development of a surface grid generator which can develop 
an optimal representation of a surface from noisy data, e.g., medical
imaging. An engineering application would be to accelerate the mesh generation
process for fluids simulations by automatically generating surface meshes
that capture local geometry.
