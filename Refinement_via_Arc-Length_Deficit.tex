% Refinement via Arc-Length Deficit
\section{Refinement via Arc-Length Deficit}
Mentioned above, a way of determining ``how well'' a discretization 
approximates a curve is to consider the difference between arc lengths. 
This is another way to determine ``how well'' a $discretization$ captures curvature. Locally, it is important for each segment in the discretization to represent the local geometry present in the curve. If a segment is to be subdivided in order to improve the discretization, then it should be subdivided 
effectively/efficient locally.
%in a way that is most effective/efficient locally. 
Also, if the purpose of the refinement process is to minimize the actual arc length minus the discrete arc length, then an optimization problem can be formed where an objective function is minimized as the combined length of the segments in the discretization approach that of the curve.

Let $C(u)$ be a parameterized curve, and $D$ be a discretization of the 
curve comprised of $n_t$ points, $P_i : i \in \{1,...,n_t\}$, and 
segments, $S_j : j \in \{1,...,(n_t-1)\}$. Segment $S_j$ is defined by two successive parameterization values, $u_j$ and $u_{j+1}$. If $L(S)$ is a function that calculates the length of a segment in $(x,y)$ space then the optimization problem can be stated as:
\begin{eqnarray*}
\begin{array}{rcl}
\underset{u_i}{\text{minimize}} \ O & = & -\sum{_{j=2}^{n_t-1}L_j} \\
\text{subject to} \ u_1 & = & a \\
u_1 & < & u_2, \\ 
u_2 & < & u_3, \\
& \vdots & \\
u_{n_t-2} & < & u_{n_t-1},\\ 
u_{n_t} & = & b.
\end{array}
\end{eqnarray*}

The resulting optimization problem is a mixed integer linear programming 
problem if both the parameterization values and number of interior points 
on the curve are unknown.  Mixed integer linear programming problems can 
be solved by a variety of standard techniques (e.g., MINTO~\cite{minto}, 
MOSEK~\cite{mosek}, or SYMPHONY~\cite{symphony}).  However, such problems 
are, in general, NP-hard.  
%Alternatively, the optimization problem would 
%need to be solved several times with an increasing number of interior 
%points until the combined length of the segments in the discretization 
%converged to within a desirable tolerance.  This is also a 
%computationally-expensive solution.  
Although the analysis and computation are straightforward for a fixed 
number of interior points, one would need to specify {\it{a priori}} this 
number, which is impractical.

Instead, in order to derive a practical algorithm that controls the 
number of interior points in the discretization, we add user-defined 
bounds on the distance between points in the discretization.  This turns 
the optimization problem into a linear programming problem.  To this end, 
let $e$ and $m$ represent user-defined lower and upper bounds, 
respectively.  In addition, let $\hat{S}_i$ denote the $i^{th}$ segment 
in non-parametrized space, i.e., $\hat{S}_i = P_{i+1}-P_i$.  The bounds 
are easily worked into the set of constraints, where $P_i$ represents a 
point in non-parametrized, $(x,y)$-space as follows: 
\begin{eqnarray*} 
\begin{array}{rcl} 
P_1 & = & \alpha,\\ 
e \leq & \hat{S}_1 & \leq m \\ 
e \leq & \hat{S}_2 & \leq m \\ 
e \leq & \vdots & \leq m \\ 
e \leq & \hat{S}_{n_t-1} & \leq m \\ 
P_{n_t} & = & \beta. 
\end{array} 
\end{eqnarray*}

The definition of lower and upper bounds for the distance between points 
implicitly defines an upper and lower bound for the number of interior 
points. The implicit definition would be in the form of an 
over-constrained problem where solutions did not exist for too many or too 
few points. For instance, too many points could not satisfy the 
minimum-distance set of constraints and too few points could not satisfy 
the maximum-distance set of constraints. However, explicitly determining 
these bounds for $n_t$ would prove difficult. For example, it could, 
involve repeatedly sampling the curve to determine the maximum number of 
e-length segments and the minimum number of m-length segments. This would 
be possible but is inefficient. Another option is to estimate the number 
of points needed \cite{cuilliere97}. However, if $n_t$ is to be estimated 
then the discretization is not guaranteed to globally optimal. Therefore, 
a global optimization problem, while possible, is not 
very practical in this case. One of the aims of this work is to 
accelerate the generation of suitable grids for simulation; and moving the 
bottleneck for grid generation to the lowest level in the grid generation 
hierarchy just increases the amount of time required to generate a grid. 
The above method does however represents a solution to the problem of 
generating automated, optimal edge-grids.

Others have attempted dynamic programming methods for generating ``optimal'' discretizations for digital curves \cite{horng02}. However, in general this should prove no more effective than any of the approaches mentioned above. It is true that the problem of generating a discretization to accurately represent a curve exhibits optimal substructure—which is defined where ``...an optimal solution can be constructed efficiently from optimal solutions to its subproblems.'' \cite{cormen01} However, the number of distinct subproblems available that represent an optimal solution at a defined error bound can be infinite. Therefore, instead of trying to find an optimal number of nodes required for an optimal discretization (which seems very inefficient), the proposed algorithm will use a divide-and-conquer (recursive) approach to generating an ideal discretization relative to a given tolerance. The combination of the optimized segments represents an optimal discretization for the entire curve.

This would generate an optimal solution using two segments to represent the entire curve -- which can be stated another way as maximizing the perimeter of the triangle formed by the existing segment and the two new segments. The above optimization problem could then be applied recursively to each new segment with $n_t=3$. This process breaks the task of optimizing a discretization for an entire curve into optimizing a simple discretization for smaller section of the curve with the following algorithm. The optimization algorithm described above with $n_t=3$ for a given segment is:

\begin{algorithm}
\caption{Optimization Algorithm with $n_t=3$}\label{alg:localoptimize}
\begin{algorithmic}[1]
\State $n_t = 3$
\State $u_i : i \in \{1,2,3\}$
\State $u_1 $ and $u_3$ define the segment $S_{1,3}$
\Procedure{Local Optimization}{$S_{1,3}$}
  \State $L(S_{1,3}) =\textnormal{ length of segment}$
  \State Place interior point $u_2$ to maximize $L(S_{1,2})+L(S_{1,3})$ within $tolerance$
\EndProcedure
\end{algorithmic}
\end{algorithm}

The would be applied for each segment in the discretization by:

\begin{algorithm}
\caption{Optimization Algorithm for Discretization}\label{alg:discreteoptimize}
\begin{algorithmic}
  \State $D(S_j) : j \in {1}$
  \State push $S_1$ into $list$ \Comment{$list$ is $queue$ if breadth-first, $stack$ if depth-first}
  \While{$list$ is not empty}
    \State pop $S_j$ from $list$
    \If {$S_i$ is optimal}
      \State do nothing
    \Else
      \State optimize $S_j$ with \Cref{alg:localoptimize}
      \State push $S_{i,i+\frac{1}{2}}$ into $list$
      \State push $S_{i+\frac{1}{2},i+1}$ into $list$
    \EndIf
  \EndWhile
\end{algorithmic}
\end{algorithm}

The above algorithm is straightforward and is often referred to as 
adaptive refinement or enrichment (see above). Also, since the 
discretization of the curve exhibits optimal substructure, the starting 
point to the optimization algorithm and the method of refinement or 
enrichment are irrelevant to the extent to which they prevent an optimal 
solution from being generated. However, they obviously contribute 
to the efficiency of the algorithm.

Now to define the optimization part of the above algorithm: Divide and 
conquer can be considered to be based on multi-branched recursion. The 
objects to be constructed at the end of the recursion are the smallest set 
of segments that approximates the arclength of the curve to a defined 
precision. This is not quantifiable without {\it{a priori}} knowledge 
of the arclength of the curve. As discussed above, calculating the length 
of a curve for this application is impractical. So how can the 
``goodness'' of a discretization be measured?  At each step, the 
discretization will be refined on each segment by locally optimizing an 
objective function analogous to the one developed above. In order to 
minimize each segment's objective function, arclength deficit ($ALD$), 
then a point $P$ has to be placed on the curve on the segment such 
that the new sum of the arc lengths is changed maximally. Since this 
entire project is to be done without calculating derivatives (see reasons 
above), the optimization scheme chosen here is not given access to 
derivative information either. Since the objective function is a 
non-negative planar curve, $(O: C \rightarrow ALD)$, any line-searching 
method of optimization could be used. However, the method cannot have any 
requirements on differentiability due to the possibility that the 
derivative of $O$ could be discontinuous.

The golden section search method is implemented here since it meets all of 
the above criteria and has the possibility to converge superlinearly 
\cite{brent73}--unlike the bisection method.  Alternatively, a pattern 
search~\cite{hopspack}, simplex~\cite{dantzig1,dantzig2}, or interior 
point method~\cite{karmarkar} could be used.  If the length of each 
segment 
locally approaches the portion of the curve it represents (i.e., the 
local objective function is 
minimized), then the global length of the discretization approaches the 
global length of the curve (a restatement of the property of optimal 
substructure). Also, since the optimization algorithm for each segment is 
only concerned about the portion of the curve it represents, then this 
method exhibits scale-independence, which was one of our requirements.

Recursive algorithms require stopping criteria. In this case the stopping 
criteria should not permit the method to infinitely subdivide the curve. 
For instance, the aforementioned minimum and maximum segment lengths can 
be used (and were implemented here). Even though using a minimum edge 
length would prevent the infinite subdivision of the curve, another 
criteria is needed such that the minimum segment length is not needed to 
satisfy the criteria. This stopping criterion could be in the form of a 
delta-segment length. That is, if the new segments' combined length is 
below a defined fraction larger than the existing segment then it should 
not be subdivided. This is a ``pure-greedy'' method of subdivision, in 
that it does not consider the rest of the ``solution'' when deciding to 
stop. One problem with this set of stopping criterion is immediately 
apparent: the ``large'' segments could potentially not be subdivided 
because locally it is not justified--even if the subdivision of the large 
segment would cause a global change in the length of the curve that is 
significant. This value, global delta-segment, would have to be smaller 
than the one used locally for each segment; otherwise it would have no 
effect. Therefore, an additional criterion is needed to determine if a 
segment should be subdivided: if the total change in length of the 
discretization would be changed by a defined fraction then it should be 
subdivided. The addition of this last subdivision criterion makes the 
method ``less-greedy''. This set, minimum segment length, maximum segment 
length, local delta-segment, and global delta-segment define a robust, 
minimum set of criterion needed for generating an optimum solution to the 
problem of representing a curve via arclength deficit.

