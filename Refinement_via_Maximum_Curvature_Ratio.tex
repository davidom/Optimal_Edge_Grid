\section{Refinement via Maximum Curvature Ratio/Deviation}
{\bf{DAVE:  Cite this statement.  I commented out the text.  Consider
adding to another section if appropriate.}}

%One could, for instance, find the maximum value for curvature ratio on 
%the segment to determine whether or not to further subdivide the segment.  
%This is an example of ``deviation-based'' refinement.  For a given segment, 
%the maximum curvature ratio is found where the perpendicular distance 
%between the segment and the curve is maximum.  This might not correspond 
%to the point where the combined arc length of the two new segments is 
%maximally different from the current configuration.  For instance 
%consider the triangle, (\textit{A}, \textit{B}, \textit{C}) with sides of 
%length following the Pythagorean triple (5,12,13) and therefore perimeter 
%of 30.

%\begin{figure}[h!]
%  \center{\includegraphics
%    {Figures/CurvatureRatioTriangles.eps}}
%  \caption{\label{CurvatureRatioTriangles} Caption}
%\end{figure}

%\noindent Additionally consider the triangle (\textit{A}, \textit{B}, 
%\textit{D}) formed with the same base (12) and height (5) as the other 
%triangle.  The perimeter of this triangle, (\textit{A}, \textit{B}, 
%\textit{D}) is 27.62. Using point \textit{C} to increases the length of 
%the segment locally by 150\%, while using point \textit{D} increases the 
%length of the segment by 130.1\%.  Both of these triangles have the same 
%curvature ratio since their bases and heights are identical.  However, as 
%shown the perimeter can vary a not-insignificant amount without changing 
%the curvature ratio of a segment.  Therefore, the curvature ratio is not 
%sufficient to determine whether or not the arc length of the 
%discretization is approaching that of the curve—only that the distance 
%between the discretization and the curve is approaching some value 
%locally.  Using only the curvature ratio, the choice between point 
%\textit{C} and \textit{D} in Figure-\ref{CurvatureRatioTriangles} are 
%equal.  The potential error would get worse if the point \textit{C} were 
%moved further left--demonstrating that the curvature ratio is not always 
%a good indicator of discretization accuracy for curves that aren’t 
%``well-behaved'' between discrete segments.

